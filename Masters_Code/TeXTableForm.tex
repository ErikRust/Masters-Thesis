
\documentstyle{article}
\begin{document}
\textheight25cm
\textwidth16cm

\author{Matthias Schunter\\ Teichstr. 1\\[0.3em] D-2900 Oldenburg}
\title{{\tt TeXTableForm.m} --- A package for converting Mathematica 2.0 Lists 
into LaTeX Tables}
\maketitle
\vfill
\tableofcontents
\vfill
\vfill
\clearpage
\section{What it does}
The TeXTableForm package generates LaTeX input from Mathematica lists. 
It may be used to typeset large amounts of data 
(e.g.\ with supertab.sty).

This Version is 1.0 from Januar 28,1993 
and it is public--domain. Changed versions must not be distributed 	
without prior written permission.

Send bug reports and suggestions to:

\centerline{\verb+schunter@informatik.uni-hildesheim.de+}.

\section{How it works}
\verb+TeXTableForm[list,n]+ generates a LaTeX-table
each sublist of list is put into one litle row. 
These rows are ordered in n big columns. 
\verb+list+ may be nested up to level 2.

TeXTableForm[list,n,\"filename\"] writes the output to the according file

\section{Options}
\verb+TableBegin->"def"+ sets the table definition to def.
The identifier \verb+columns+ may be used to insert the number of big columns;
The identifier \verb+width+ may be used to insert the width of each column. 
Default is: 
\begin{verbatim}
TableBegin -> "\begin{tabular}{|*{columns}{*{width}{|l}|}|}\n"
\end{verbatim}

\verb+TableEnd->"def"+ sets the text wich is included at the end of the table.
Default is \verb+TableEnd -> "\end{tabular}"+. This Option is useful
if you want to use the supertabular environement.


\section{A short example}
The Mathematica input:

\begin{verbatim}
In[1]:= <<TeXTableForm`;

In[2]:= t=Table[i,{i,1,22}];

In[3]:= TeXTableForm[t,5,"tab1.tab"];
\end{verbatim}

The result is shown in Table~\ref{tab1}.
This table was produced with:
\begin{verbatim}
\begin{table}
\begin{center}
\input tab1
\end{center}
\caption{Our first example}\caption{tab1}
\end{table}
\end{verbatim}
\begin{table}
\begin{center}
\begin{tabular}{|*{5}{*{1}{|l}|}|}
$1$ & $6$ & $11$ & $16$ & $21$ \\
$2$ & $7$ & $12$ & $17$ & $22$ \\
$3$ & $8$ & $13$ & $18$ &   \\
$4$ & $9$ & $14$ & $19$ &   \\
$5$ & $10$ & $15$ & $20$ &   \\
\end{tabular}
\end{center}
\caption{Our first example}\label{tab1}
\end{table}

Our next example is a little more difficult:
\begin{verbatim}
In[1]:= <<TeXTableForm`;

In[2]:= t=Table[{i,Sqrt[i]},{i,1,22}];

In[3]:= TeXTableForm[t,3,"tab2.tab"];
\end{verbatim}

The result is shown in Table~\ref{tab2}.
It was produced by:
\begin{verbatim}
\begin{table}
\begin{center}
\input tab2.tab
\caption{Our second example}\label{tab2}
\end{center}
\end{table}
\end{verbatim}
\begin{table}
\begin{center}
\begin{tabular}{|*{3}{*{2}{|l}|}|}
$1$ & $1$ & $9$ & $3$ & $17$ & ${\sqrt{17}}$ \\
$2$ & ${\sqrt{2}}$ & $10$ & ${\sqrt{10}}$ & $18$ & $3\,{\sqrt{2}}$ \\
$3$ & ${\sqrt{3}}$ & $11$ & ${\sqrt{11}}$ & $19$ & ${\sqrt{19}}$ \\
$4$ & $2$ & $12$ & $2\,{\sqrt{3}}$ & $20$ & $2\,{\sqrt{5}}$ \\
$5$ & ${\sqrt{5}}$ & $13$ & ${\sqrt{13}}$ & $21$ & ${\sqrt{21}}$ \\
$6$ & ${\sqrt{6}}$ & $14$ & ${\sqrt{14}}$ & $22$ & ${\sqrt{22}}$ \\
$7$ & ${\sqrt{7}}$ & $15$ & ${\sqrt{15}}$ &   &   \\
$8$ & ${2^{{3\over 2}}}$ & $16$ & $4$ &   &   \\
\end{tabular}
\end{center}
\caption{Our second example}\label{tab2}
\end{table}

\end{document}

