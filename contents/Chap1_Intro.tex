\chapter{Introduction} \label{chp:1}
%%%%%%%%%%%%%%%%%%%%%%%%%%%%%%%%%%%%%%%%%%%%%%%%%%%%%%%%%%%%%%%%%%%%%%%
An ever increasing amount of knowledge on control and regulation of metabolic pathways is generated everyday \cite{Boogerd2007, Khalil2015}. This knowledge is stored by means of mathematical models of biological phenomena. Although all modeling aims stem from a need for the quantitative understanding of natural phenomena, these models can originate from various walks of scientific inquiry. Therefore, in this study the focus will be on areas of systems biology, biochemistry and bio-engineering. These models are more specifically geared towards understanding, describing and ultimately predicting the outcomes of perturbations on various collective or specific parts of biochemical systems. With all of the complexities to living systems it is understandable that the amount of collective knowledge is so truly vast. With each new publication, a new, revised, reformatted, reinterpreted or re-imagined model is set forth and soon enough, a problem of scale is encountered. When it comes to the continuous interpretation of these models, humans simply do not have the manual capabilities to deal with such large amounts of data. Thankfully systems biology approaches have been identified as being particularly useful towards more fully exploring living system models within this ever growing "omics" era as discussed in chapter \ref{chp:2} \cite{Kell2006, Ho2008, Lee2006, Knudsen2004, Bleicher2003, Steuer2003}. 

The synthesis of knowledge, from experimentally repeatable observations to mathematical equations known as the formalism, is referred to as the modeling process. The derivation and interrogation of such mathematical models have been of central interest to various fields of biological research, to be elaborated upon in chapter \ref{chp:2}. For the main part two distinct stages are involved in the modeling process, creation and validation. Together these form a continues development cycle, adapting and improving the model functions, until the desired degree of accuracy and predictability is achieved. The construction of a biochemical model follows from first the identification to the description and quantification of a myriad of parameters and time dependant variables. For example the stoichiometry, interaction mechanisms, reaction rates, binding constants and the effects of changes in \pH upon all other variables. Physiological relevance is obtained by populating these variables with parameter values from results that were obtained experimentally. The validation of the model then involves the ability of the model prediction to accurately reflect the experimentally determined results. Robustness is then a term that is used to describe the range of variability across which this model holds accurate predictive values.

Once a model has been validated it can be formatted in the standard biological markup language (\gls{sbml}) for storage on-line in biological model repositories such as \href{https://www.ebi.ac.uk/biomodels-main/}{BioModels} and \href{https://jjj.bio.vu.nl}{JWSOnline}. These model repositories give rise to interesting investigative possibilities, as this information is openly available and can be interrogated and analyzed by any method at the researcher's disposal. 

One such method will be discussed in this thesis in an attempt to disprove a relationship between the flux control coefficient and disequilibrium ratio of an enzyme catalyzed reaction within metabolic pathways. Considering the hypothetical case of a reaction at equilibrium, neither an increase, nor decrease in enzyme activity is seen to greatly influence the overall rate of the system. Inversely then, would a reaction rate achieved far from equilibrium necessarily greatly influence the network by the addition or removal of enzymes? This is an interesting question, especially as it has great relevance on faster ways to determining whether an enzyme is a suitable drug target. In the brewing and processing industries, faster determination of key enzyme levels can lead to large scale savings and gains in efficiency. 

By utilising modern internet communications protocols and scripting software, a method is developed in order to test whether a relationship between the flux control coefficient and disequilibrium ratio can be observed, not only within a single network, but across many hundreds of models. The disequilibrium ratio of a reaction will serve as quantification for the thermodynamic state at which the enzyme control is determined. The method for automated disequilibrium calculations was developed and control calculations was done by means of metabolic control analysis (\gls{mca}) as is discussed in chapter \ref{chp:3}. 

The relevance of a study such as this lies in the inherent understanding of the underlying mechanisms of control with various abstract future implications. A more near future outcome however is in the application of automated interrogation approaches to biological models for various other questions than this one. As stated by \citeauthor{Knudsen2004}, "Perhaps a computational investigation can reveal information relevant to applications in pharmaceutics and medical therapies." \cite{Knudsen2004}.

 

