\chapter{Introduction} \label{chp:1}
%%%%%%%%%%%%%%%%%%%%%%%%%%%%%%%%%%%%%%%%%%%%%%%%%%%%%%%%%%%%%%%%%%%%%%%
From the first laws of motion and thermodynamics the underpinnings of the constituent energy-matter existing within our universe is described. Set in motion by forces, as dictated by the second law of thermodynamics, predicating that the entropy of the universe is ever increasing. The end result of this increase inevitably leading to an ultimate state of maximal disorder \cite{Planck1905}. In the meanwhile, energy in our universe is contained within a gradient leading from low to high states of entropy. Life has been said to exist as an intermediate stability within these gradients of energy throughout the universe \cite{Schneider1994, Prigogine1998}. It therefore follows that regulation of living systems have developed regulatory properties based on thermodynamic properties of their state. 

With entire fields of systems biology, biochemistry and bio-engineering, geared towards describing, understanding and ultimately influencing living systems. A wealth of information on enzyme kinetics have been gathered over the years \cite{Boogerd2007, Khalil2015}. With ever increasing applications, yielding mixed results depending on the overall goal, systems biology approaches have been shown to be crucial in understanding biochemical systems in the growing "omics" era  \cite{Kell2006, Ho2008, Knudsen2004, Bleicher2003, Steuer2003}. 

The synthesis of experimental and theoretical investigations is seen in the accumulation of mathematical models, attempting to describe observed biological phenomena. In general, biological models are constructed by identifying and describing structures and reaction processes involved in the system under investigation. For example stoichiometry, interaction mechanisms, reaction rates, rate constants and other parameters are to be quantified. This is done by encapsulating these properties in a mathematical formalism, commonly referred to as a biological model. Many models obtain parameter values from experimental data, leading to various degrees of physiological relevance. Once a model has been created and validated it can be transcribed into the standard biological markup language (\gls{sbml}) and stored on-line in biological model repositories such as \href{https://www.ebi.ac.uk/biomodels-main/}{BioModels} and \href{https://jjj.bio.vu.nl}{JWSOnline} for reuse. These model repositories give rise to interesting investigative possibilities, as this information is openly available and can be interrogated and analyzed by any method at the researcher's disposal. 

In this paper I present a method, integrating metabolic control analysis, thermodynamic state measurements and data-mining techniques to gather knowledge on the relationship between fundamental thermodynamic properties of a reaction and the enzymatic regulatory portion involved. The disequilibrium ratio of a reaction served to quantify the thermodynamic state at which the enzyme control was determined. Control calculations was done by means of metabolic control analysis (\gls{mca}) as discussed in chapter \ref{Methodology}. A second portion will be dedicated to current strategies towards the problem of identification of regulatory enzymes, given sparse information on kinetics. Reference to the comparative validity of \gls{mca} as a tool towards this aim of will be handled concurrently. The focus on aspects of thermodynamics and enzyme kinetics involved in biochemical systems.

A need for an approach of the automated and scalable kind seems in order, given the vast amounts of biochemical models available on data-repositories around the world \citep{Kitano2002}. Also given the amount of unknowns surrounding the control structures of living organisms. As stated by \citeauthor{Knudsen2004} perhaps a computational investigation can reveal information relevant to applications in pharmaceutics, medical therapies and more.
