\chapter{Introduction} \label{chp:1}
%%%%%%%%%%%%%%%%%%%%%%%%%%%%%%%%%%%%%%%%%%%%%%%%%%%%%%%%%%%%%%%%%%%%%%%
Large scale genomics and metabolic studies are generating more data than can be managed by the human mind alone \cite{Khalil2015, Boogerd2007, Kell2006}. Increasingly, this knowledge is stored by means of mathematical models generated from observed biological phenomena. Most modeling aims stem from a need for the quantitative understanding of these phenomena. Models can originate from various walks of scientific inquiry. Such as systems biology, biochemistry, pharmaceutics and bio-engineering. Often these models are more specifically geared towards understanding, describing and ultimately predicting the outcomes of perturbations on various collective or specific parts of biochemical systems \ref{chp:2}. Utilising these models with an automated computational analysis method, reduces the challenge of data scale \cite{Ho2008, Lee2006, Knudsen2004, Bleicher2003, Steuer2003}. 

The synthesis of knowledge, from experimentally repeatable observations to mathematical equations known as the formalism, is referred to as the modeling process. The derivation and interrogation of such models have been of central interest to various fields of biological research, to be elaborated upon in chapter \ref{chp:2}. For the main part two distinct stages are involved in the modeling process, creation and validation. Together these form a continues development cycle, adapting and improving the model functions, until the desired degree of accuracy and predictability is achieved. The construction of a biochemical model follows from first the identification to the description and quantification of a myriad of parameters and time dependant variables. For example, the stoichiometry, interaction mechanisms, reaction rates, binding constants and $\pH$. Physiological relevance is obtained by populating these parameters with values from results that were obtained experimentally. The validation of the model then involves testing towards the ability of the model to acurately predict experimental results. Robustness is another term used to describe the range of variability across which this model holds validity.

Once a model has been validated it can be formatted in the standard biological markup language (\gls{sbml}) for storage on-line in biological model repositories such as \href{https://www.ebi.ac.uk/biomodels-main/}{BioModels} and \href{https://jjj.bio.vu.nl}{JWSOnline}. These model repositories give rise to interesting investigative possibilities, as this information is openly available and can be interrogated and analyzed by any method at the researcher's disposal. 

One such method will be discussed in this thesis in an attempt to test a possible relationship between the flux control coefficient and disequilibrium ratio of an enzyme catalyzed reaction within metabolic pathways. This is an interesting question as it contains relevance towards faster ways of detecting relavent enzymes for drug targeting.A further application can be foun in the brewing and processing industries, where faster determination of key enzyme levels can lead to large scale savings and gains in production efficiency. 

By utilising modern internet communications protocols and scripting software, a method is developed in order to test whether a relationship between the flux control coefficient and disequilibrium ratio can be observed, not only within a single network, but across many hundreds of models. The disequilibrium ratio of a reaction will serve as quantification for the thermodynamic state at which enzyme control is determined. The method for automated disequilibrium calculations was developed and control calculations was done by means of metabolic control analysis (\gls{mca}) as is discussed in chapter \ref{chp:3}. 

The relevance of a study such as this lies in the inherent understanding of the underlying mechanisms of control. Automated interrogation approaches to biological models can also be used towards questions with. As stated by \citeauthor{Knudsen2004}, "Perhaps a computational investigation can reveal information relevant to applications in pharmaceutics and medical therapies." \cite{Knudsen2004}.

The aims and objectives were as follow.
Aim 1 - Literature review.
	Objectives: Current possible automated approaches towards disequilibrium and flux 				control coefficient determinations.
				Views on possible disequilibrium ratio to flux control coefficient relationships.
				Frameworks on data mining.
				Mathematica documentation.

Aim 2 - Retrieve data from online repositories.
	Objectives: Retrieve repository API endpoint information
				Construct function utilising API endpoint.
				Manage model information in order to maintain traceability in a parallel environment.

Aim 3 - Calculate flux control coeficient and disequilibrium ratios.
	Objectives: Filter models based on defined criteria towards calculations.
				Calculate flux control coeficients.
				Calculate disequilibrium ratio.

Aim 4 - Vizualise and describe data
	Objectives: Normalise and plot data in a two dimensional graph.
				Train neural networks on data.
				Compare computational prediction with visual observations.

Aim 5 - Write it down.

 

