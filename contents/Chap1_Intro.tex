\chapter{Introduction}
\label{chp:1}
%%%%%%%%%%%%%%%%%%%%%%%%%%%%%%%%%%%%%%%%%%%%%%%%%%%%%%%%%%%%%%%%%%%%%%%
Two fundamental elements of biochemical modelling, stoichiometry and rate equations can be seen to parallel the underlying laws of nature. 

Biochemical systems' stoichiometry, in simplest form, echoes the conservation of matter and energy as defined by the 1_{st} Law of Thermodynamics. Rate equations then further describes this process and the regulation thereof, as it varies from state to state. This change in state is driven by entropy as termed by the 2_{nd} law of thermodynamics wherein natural systems have a tendency towards a highest degree of disorder. In biochemistry we refer to this as the equilibrium condition \cite{Prigogine1998}. 

In an attempt at capturing experimental observations, on these state changes, biochemical models is considered an indispensable tool. A model summarizes the current understanding of a specific metabolic pathway into a reusable format. Biochemical models are able to obtain various degrees of accuracy and realism, defined on an as needed basis, by incorporating experimental results at relevant physiological conditions. These observations are incorporated through the use of parameters and augmentative functions as is discussed in chapter \ref{chp:3}. 

The end result is ideally a model detailing all of the above succinctly and elegantly. In practice however, the level of detail needed remains in question. As either detail is not needed or feasible within certain time frames for the specific question at hand. As such, little emphasis have been placed on having ideally Thus we embark on a journey of exploration through models that have been accumulating on model repositories through the years. In an attempt to quantify the current understanding of enzymatic flux control properties.




"In the beginning, there was light". A sentence that, according to scripture, brought life into existence.

Casting aside any religious debates for now, let us rather entertain a simple thought experiment originating from this sentence. In the early universe, gravity pulls clouds of vapour and dust together, swirling inwards, turning, ejecting larger massed objects outwards into space and maintaining smaller atomic massed elements in larger concentrations towards the centre, much like a centrifuge as described by various biochemistry textbooks. As this dense cloud of mostly hydrogen, helium and trace amounts of arguably all other "natural" elements gravitates towards centre of mass. atoms are forced infinitesimally close together,  pressure and heat can no longer be sustained and in an instant\ldots Mass is turned into energy as atoms fuse together and a sentence rings true; "Let there be light". Travelling across the universe as a carrier state, packets of energy moving through space, heating matter as it spills onto, through, into and across all of existence.  

Drawn from the above, be nothing else true, one can imagine how light from our sun sustains life on earth. At this very moment energy is being shed into the universe, \gls{autotrophs} consuming said energy, constructing larger more complex molecules, by cleverly capturing photons and relaying energy to work creating order in molecules. By utilising the natural decay of the sun, as predicted by the laws of thermodynamics; creating order at it's own level at the expense of order in larger systems within the hierarchy as discussed by \citeauthor{Bertalanffy1972}. In this way counteracting not only it's own entropic increase, but that of the entire diversity of life feeding from it. In our case both directly and indirectly, through fruits, vegetable, meats, fish dairy and essentially all else that we consume and rely on. These in turn again are all dependant upon the energy derived from our sun in own right, continuously allowing life to exist within a "sweet spot" of an energy gradient. This process can essentially be viewed as open systems in turn interacting again with various other open systems, construction from destruction. 

Now this of course is not a novel concept, but it does lead to the discussion of fundamental laws and interactions governing these microscopic actions that will be discussed in more depth. The first and second Laws of thermodynamics for instance. As the first law states; No matter can be destroyed or created, however changes of state is permitted. The first law of thermodynamics is therefore taken to frame the environment within which life occurs, as such this law will not be discussed further within the context of this text. Of more importance to the topic at hand, is the second law of thermodynamics, as it had been stated to be a driving force of life \cite{Schneider1994}. Attention will also be given to \gls{mca} as this will for the framework upon which enzyme control is analysed. Another law worth mentioning is the law of mass action as formalized by \citeauthor{Waage1986} and later derived from thermodynamic principles by \citeauthor{VantHoff1884}. From the mass action law, a more general power law has been established by \citeauthor{Savageau1969a}, this too will be considered. All of these are examined in the spirit of understanding various methods of biochemical modelling, in an attempt to derive a computational function for the purpose of comparing the disequilibrium ratio to the flux control co\"efficient of various reactions within different models. This function will be applied to models meeting criteria as specified in the methods section.
A need for an approach of this kind seems in order, given the vast amounts of biochemical models available on data-repositories around the world \citep{Kitano2002}. Also given the amount of unknowns surrounding the control structures of living organisms. As stated by \citeauthor{Knudsen2004} perhaps a computational investigation can reveal information relevant to applications in pharmaceutics, medical therapies and more. This specific approach aims to relate the control exerted by enzymes upon their own-flux to the disequilibrium ratio of the reactions they catalyse. Otherwise stated, a quantitative investigation into the conditions arising in \gls{steady-state} enzyme catalysed reactions in order to ascertain whether information on the distance away from equilibrium (thermodynamic aspects) can yield insights into the control over flux of said enzymatic reaction (enzyme kinetic aspects). The function in discussion will therefore need to calculate the disequilibrium ratio as well as the flux control co\"efficient for reversible reactions from models retrieved from on-line data-repositories. The flux control co\"efficient is chosen as it has proven to be of importance with regards to various disciplines as discussed in the literature review. This function needs to be robust and general enough to be able to handle various models that were created for specific conditions that, from the computational perspective remains unknown.
