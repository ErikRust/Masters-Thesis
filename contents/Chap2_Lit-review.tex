\chapter{Literature Review}
\label{chp:2}
As mentioned in chapter \ref{chp:1} there are two main aspects contributing to the development of this thesis; thermodynamics, in the form of the disequilibrium ratio, served as an independent variable. Furthermore the flux control co\"eficcient, derived from \gls{mca}, served as our variable in question. In the next section therefore, let us investigate the origin as well as current understanding of these two variables in literature.

\section{\gls{mca}}
The implication and applications of \gls{mca} have been far reaching. Bringing about new insights and developments to the domains of Biochemistry, Biomedicine, Biotechnology, Physiology, Pharmaceuticals, Genetics, Botany, Immunology, Agriculture and more \citep{Kacser1981, Sorribas1995, Holms1996, Cornish1999, Boren2002, Cascante2002, Olivier2004, Weselake2008}. With an impact on such a spectrum of disciplines an \gls{mca} could prove useful in the creation of a general function for analysing model control structures. What follows is a brief introduction to the wonderful world of \gls{mca} as well as a look at current applications thereof.

Setting off on this journey of literature exploration an initial point of departure is required. An anchor, so to speak, arrived in the form of work by Prof. \citeauthor{Hofmeyr2001}, in which he suggests starting at the beginning of it all. Two independent parties finding themselves developing similar theories on the control of metabolic networks \citep{Rapoport1974, Kacser1973,Hofmeyr2001}.

This formalism brought about advances in biological understanding as finally biologists had a framework within which to incorporate both structural and kinetic information. Investigations and understandings of biological processes could therefore occur in a less isolated \textit{i.e.} more holistic manner. This was an ideal expressed by \citeauthor{Kacser1968} to move "\ldots beyond the mere cataloguing \ldots of biochemical details." and towards defining and understanding systems as a whole. A desire echoed in the development of Systems Theory and one still applicable today. \citeauthor{Kacser1968} further  illustrates the earliest \gls{mca} intuition for \gls{mca} on the basis of analytical investigation methods. Two methods described; one in which a system is studied by investigating isolated portions and the second in which a system is immobilized in a specific state by maintaining certain variables constant and altering others \citep{Kacser1968}. The articulation of these two methods can be seen as constituent parts of \gls{mca}, where the isolated systems perspective still forms a basis for \gls{mca}; Through enzyme kinetic results largely driven by \textit{in vitro} experimental work on isolated enzymes \citep{Hynne2001, Boren2002}. The second part will be discussed within the context of the \gls{steady-state}. 

As a subset of sensitivity analysis, \gls{mca} is concerned with quantitatively determining effects upon a system, brought about by alterations in constituent parts of the system, what is known as a perturbations. In the context of \gls{mca} the system is kept at \gls{steady-state}, a dynamic equilibrium state whereby influx and efflux of external parameters are held constant, and by altering local parameters \textit{e.g.}enzyme concentrations, $V_Max$, \textit{etc.} of this \gls{steady-state} system by a one percent change, one can relate resulting changes in systemic variables to this corresponding change in local parameter. From these types of perturbations three different co\"efficients can be derived. The flux control-, concentration control co\"efficients and elasticity values, describing the quantitative roles of various elements of the system, upon the system itself \citep{Kacser1979}. A discussion on these components follows.

\subsection{Control co\"efficients and elasticities} 
As stated above, various control co\"efficients and elasticities exists. Some of these form the basis of the relationships under investigation in \gls{mca}.

The first of these relationships is the flux control co\"efficient. The flux control co\"efficient gives a quantitative description of the effect a local perturbation \textit{i.e} enzyme concentration, brings about upon the flux through each reaction within the pathway.

\begin{equation}
C^{J}_{i}\approx\dfrac {\delta J}{J}/\dfrac {\delta e_{i}}{e_{i}}
\end{equation}

This co\"efficient relates individual effects upon flux through the system, however without constraints on the systemic flux, this relationship, in and of itself is not very descriptive of the system as a whole. This is true for all of the co\"efficients handled here, an inadequacy addressed by formal theorems later. For now, let us confine our enquiries to the co\"efficients at hand.

A co\"efficient describing the effect upon metabolite concentrations, given a partial change in enzyme concentration, is termed the concentration control co\"efficient.

\begin{equation}
C^{S}_{i}\approx\dfrac{\delta S}{S}/\dfrac{\delta e_{i}}{e_{i}}
\end{equation}

In contrast to the previous two co\"efficients, elasticities relate a change in some systemic factor to the corresponding change in the rate of the reaction, for example, the effect of reaction temperature, metabolite concentrations, pH and more. In mathematical notation the elasticity co\"efficient is defined as.

\begin{equation}
\varepsilon^{i}_{P}\approx\dfrac{\delta v_{i}}{v_{i}}/\dfrac{\delta P}{P}
\end{equation}


The first two co\"efficients, respectively termed the flux control and concentration control co\"efficients, relate local parameter changes to global variable effects, whereas the elasticity control co\"efficient relate local variable changes to effective changes in corresponding reaction rates. From these control co\"effiecients and elasticities two relationships, the connectivity and control theorems, were derived. These relationships were later extended to cases involving symbolic matrix algebra methods which are used in this thesis \cite{Fell1992,Kacser1995,Ehlde1997,Hofmeyr2001}. As mentioned earlier, the effects described by these various co\"efficients and elasticities, possess little information about the effects perturbations have upon the system as a whole. To address this, various theorems were developed in order to; relate these relationships described to one another and  set constraints upon the system within which a more holistic analysis can occur.

\subsection{Summation and Connectivity Theorems}
The summation theorem as described by \citeauthor{Kacser1979,Rapoport1974} defines the boundaries of our systems, as it sets constraints, \textit{viz.} 

\begin{equation}
\sum_{i}C_{v_i}^J=1
\end{equation}

\begin{equation}
\sum_{i}C_{v_i}^S=0
\end{equation}

Equation 2.4 describes the distribution of flux control throughout the system, with each reaction sharing in part towards the whole of flux control for the pathway. Through eq. 2.5 it can be seen that the the effect upon metabolite concentrations via enzyme perturbations is distributed throughout the system, as concentrations in one part increases with respect to parameter perturbations, those in other parts will decrease in accordance. The establishment of a quantitative description of how local control effects systemic variables, \textit{i.e.} metabolite concentrations (eq. 2.4), as well as the effect that these systemic variables in turn have upon the rates of reactions (eq. 2.5) allows one to describe holistic systemic behaviour through a combination of these theorems, a goal achieved through the connectivity theorems.

Connectivity theorems define a link between local enzyme kinetic properties and global pathway variables based on two basic assumptions. The first being on the additive properties between the products of flux control co\"effiecients and elasticities.

\begin{equation}
\sum_{i}C_{i}^J\epsilon_S^i=0
\end{equation}

The second theorem involves the summation property of the products between concentration control and elasticities. This second theorem is further subdivided into two equations. The first of which consider system species,($S_m$) differing to those of local species ($S_m$) and the second considering system species that equal local species \citep{Kacser1979, Westerhoff1984}.

\begin{equation}
\sum_{i}C_{i}^{S_n}\epsilon_{S_m}^i=0\quad \textrm{for}\quad m\ne n
\end{equation}

\begin{equation}
\sum_{i}C_{i}^{S_n}\epsilon_{S_m}^i=-1\quad \textrm{for}\quad m = n
\end{equation}


Many research activities have been geared towards discerning the so called "rate-limiting" step of a metabolic pathway, adopted from chemistry where the rate of the overall reaction is approximately determined by the slowest rate in the reaction chain. This concept serves as a useful tool in simplifying the mathematics involved in kinetic calculations of reactions, however, this oversimplification results in a loss of network accuracy when applied to more complex reaction pathways, such as those occurring within the metabolic context. The introduction of \gls{mca}, specifically the flux control summation theorem redefined the concept of "rate-limiting" steps. By defining the contribution of each reaction to the distribution of control over the system as a whole. As such control is shown to be shared amongst all participating reactions. In the case of flux control, as opposed to having a single "rate-limiting" reaction controlling flux through the entire system, this control is now spread across the reaction space by way of the flux control co\"efficient. Assuming at this point that the sum of the own-flux control co\"efficients across the system equals one, the system is said to obey the flux summation theorem \citep{Fell1992,Hofmeyr2001,Kacser1995,Ehlde1997}. As well as being central to the control analysis at hand, this flux summation theorem can be used in and automated approach at \gls{steady-state} validation as will be discussed in the methods chapter later.

Following an understanding of the concepts of \gls{mca} and more specifically the flux control co\"efficient, attention is given next to the second part of the investigation namely the thermodynamic aspects.

%%%%%%%%%%%%%%%%%%%%%%%%%%%%%%%%%%%%%%%%%%%%%%%%%%%%%%%%%%%%%%%%%%%%%%%
\section{Thermodynamics in biochemistry}

The first and second laws of thermodynamics states that; Not energy nor matter can be created or destroyed but can only be altered from one state to another. Within these constraints the total entropy of a large isolated system, over a sufficiently long time period, is said to never decrease.

The first law of thermodynamics can be thought of as laying the foundations upon which life has developed. The second law in turn then providing an explanation as to how systems progress through time. If the first law can be imagined as the structure, the second law would be deemed the driving force. 

\subsection{Entropy and $2_{nd}$ Law of thermodynamics}
With regards to the second law of thermodynamics, careful respect is given to the choice of words. Referring to the use of "isolated systems over sufficiently long time scales", as attempts have been made at disproving this law and some have even indeed been successful. 

At first, theoretical views from \citeauthor{Evans1993} disrupted the thermodynamic community, by defining that the probability distribution for entropy producing reactions opposing the $2_{nd}$ Law of thermodynamics are non-zero, however, this probability decreases exponentially with relation to system size and time \cite{Evans1993}. A relationship that can be distinguished from the formulation: $$\frac{Pr\left(\overline{\Sigma}_{t}=A\right)}{Pr\left(\overline{\Sigma}_{t}=-A\right)}=e^{At}$$

Simply shown, with $t$ and $\overline{\Sigma}$ denoting time and entropy respectively, as either value of $A$ or $t$ increases the ratio of probabilities, denoting entropy producing interactions in agreement to and in opposition of the second law of thermodynamics, exponentially increases. Meaning either the numerator ($Pr\left(\overline{\sum}_{t}=A\right)$) is undergoing an increase, while the denominator ($Pr\left(\overline{\Sigma}_{t}=-A\right)$) is undergoing a decrease. Either way the end result is the same; A decrease in entropy producing interactions which oppose the second law of thermodynamics.  
 From this, a formulation was derived for processes of a more time-dependant nature by \citeauthor{Crooks1999} in \citeyear{Crooks1999} and validated in \citeyear{Kasner2013} by \citeauthor{Kasner2013} in protein folding entropy producing interactions of RNA, where trajectories opposing the second Law of thermodynamics were observed during the mechanical perturbation of RNA molecules \cite{Crooks1999,Kasner2013}. With these exceptions in mind, the second law of thermodynamics still holds on a phenomenological level and as such the most important contributors should be discussed with regards to their applicability within biochemistry. As taken from the original treatise on thermodynamics, the second law hinges on the inability of an entropic decrease for an isolated system \citep{Planck1905}. What follows is therefore a discussion of entropy as relevant to biochemistry and the investigation under consideration.
 
Entropy; defined as a quantity representing the unavailability of thermal energy. Thermal energy within a specified system, our sun, for example. At birth it possessed a minimum entropy value and as such the maximal amount of thermal energy availability. As fusion reactions progress through time, matter is converted into packets of energy, photons ejected into space, abiding by the first law of thermodynamics as photons are formed when protons are forced so close together they join in a fusion reaction, upon which case electrons relax into new states and release this change in electromagnetic force.Thus the available thermal energy of the sun decreases by a similar quantity as is emitted. This can be viewed in another manner, as in biochemistry, where entropy refers to the inevitable push and pull towards energy equilibrium. In other words, synthesising more complex molecules from simpler molecules require the addition of energy, causing a decrease in the systemic entropy, as such, these are deemed thermodynamically unfavourable. Conversely, a spontaneous reaction occurring from the breakdown of a larger more organized and of higher energy molecules to smaller molecules, releasing energy in the process. This is considered a thermodynamically favourable reaction and as such is expected to proceed spontaneously. It is of course important to note that these are all subject to temperature and pressure as described by the first law of thermodynamics, within the biochemical context however these variables are taken to be kept constant. As a result the thermodynamic energy gradient, created from our sun, have allowed for very specific conditions to occur. Conditions that have giving rise to the developmental challenge of capturing and storing energy. Utilising the energy over longer periods of time than the instantaneous moment it arrives on our earth. Allowing for the maintenance of order away from equilibrium. Whether or not this process could give rise to the complexity of life as we observe and experience it today is a discussion outside the scope of this text, it is however clear that entropy is a large driving force of life. 
Another analogy of entropy, commonly referred to, is one in which a room has a tendency to get messy or more disordered, without the expenditure of energy. It then requires work, and therefore energy, to create order. This analogy has a further level of applicability since work can be randomly applied and, given enough time, objects will reach their designated "ordered" state. This approach however, would not be an effective one as energy will be spent aimlessly moving objects too and from areas. A more efficient approach is one in which energy is applied by moving objects from an initial state to the final state. To achieve such an act on must therefore first mentally determine where the final state would be, an action that in and of itself again requires energy. In a similar manner, biochemical processes within living organisms does not simply occur in a random fashion. In fact, reactions have shown to be controlled by various feedback structures, signalling pathways and supply/demand conditions in order to utilise available resource as efficiently as possible.

Entropic contributions to biochemistry, and in such thermodynamic contributions, are made explicit through the expressions of Gibbs free energy changes as in $\Delta G(p,T)=\Delta H+pV-T\Delta S$ where $\Delta G$ represents Gibbs free energy of a system,$p$ and $V$ relating the effects of pressure upon volume, $\Delta H$ the enthalpy, in biochemistry, referring to differences in bond energies and $T$ relating temperature in Kelvin as a scalar of entropy $\Delta S$. Through inspection of this equation, it is clear to see the contributions made by both entropy as well as enthalpy towards determining the state of the system with regards to equilibrium, in biochemistry we assume the system to remain at constant volume and pressure. In other words when $\Delta G > 0$ the system is away from equilibrium and will not proceed spontaneously without the addition of free energy, as such these reactions are termed endergonic. At $\Delta G = 0$ the system is defined as being at equilibrium when both forward and reverse reaction flux proceeds at the same rate, resulting in no net reaction flux. Negative $\Delta G$ values indicate a state away from equilibrium that will proceed without free energy addition, otherwise known as an exergonic reaction. Differing combinations of these terms give rise to various possible reaction states, the scalar property of $T$ towards $\Delta S$ can be seen as the temperature dependence of entropy. It follows however that from these thermodynamic terms one can identified a reaction's equilibrium conditions or $K_{eq}$ values experimentally, as the product over substrate concentration at $\Delta G =0$. The $K_{eq}$, product to substrate ratio at equilibrium, turns out to be a special case of what is known as the mass action ratio, as will be discussed.

\section{Mass action and the disequilibrium ratio}
Since the mid $16^{th}$ century, questions surrounding chemical interactions have been focussed on how dissimilar substances remain bound. At first this chemical driving force was termed "affinity" and later investigated by French Chemist/Physician Claude Louis Berthollet. Berthollet established a relation between the rate of a chemical reaction and the mass of the substance involved. This breakthrough was expanded upon later into the view that the degree of affinity, for substances towards one another, is affected by the properties of the substances in question. A finding that proved to have large implications upon therapeutics, as drug therapies can now be understood to affect different tissues with differing affinities. This insight later lead to theoretical links in the idea of "drug receptors" \cite{Ferner2016,Voit2015}. The first formal description of the law of mass action arrived from the minds of \citeauthor{Waage1986}. A further link within thermodynamics is found between the disequilibrium ratio, $\rho$ and Gibbs free energy, $\Delta G$ in what is known as the flux-force relationship.

\subsection{Flux-Force Relationship}
In a reaction from substrate $S$ to product $P$, the equilibrium constant $K_{eq}$ is defined, by the law of mass action, as the ratio of forward, $k_{+1}$, and reverse, $k_{+1}$, rate constants.

\begin{equation}
K_{eq}= \frac{k_{+1}}{k_{-1}}
\end{equation}

Forward, $J^+$, and reverse, $J^-$, fluxes are defined as $J^+ = k_{+1}[S]$ and $J^- = k_{-1}[P]$ where multiple substrate and product reactions are incorporated via the multiplication of metabolite concentrations.
The Gibbs free energy change, $\Delta G$, can then be expressed in terms of the logarithm of the ratio of forward and reverse fluxes termed the flux force relationship.

\begin{equation}
\Delta G = -RTLn(\frac{J^+}{J^-})
\end{equation}

At equilibrium conditions, $\Delta G = 0$ leading to $\frac{J^+}{J^-} = 1$ and therefore $J^+=J^-$ indicating how, at equilibrium, the forward and reverse fluxes are equal to one another \citep{Beard2007}. 

Taking 
\begin{equation}
\frac{J^-}{J^+} = \frac{k_{-1}[P]}{k_{+1}[S]}
\end{equation}
taking the multiplicative inverse of equation 2.9 and substituting it, as well as $\Gamma = \frac{[P]}{[S]}$ into equation 2.11 yields
\begin{equation}
\frac{J^-}{J^+} = \frac{\Gamma}{K_{eq}} = \rho
\end{equation}
 This is also known as the disequilibrium ratio.

\section{Power Law}
As stated in his three part article on the power law and applications thereof in biochemical systems analysis \cite{Savageau1969a,Savageau1969,Savageau1970}.

Concentration control matrix calculation
$\overline{C}^s=-LM^{-1}N_R$

\subsection{Current applications of Flux control co\"efficients}


