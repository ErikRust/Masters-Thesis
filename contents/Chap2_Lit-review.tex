\chapter{Literature Review} \label{chp:2}
This chapter ties together the theoretical aspects of thermodynamics, metabolic control analysis, biochemical modelling and how they relate to understanding the possible relationship between flux control coefficients and the thermodynamic disequilibrium ratios. The thermodynamic state of a reaction, characterised by the disequilibrium ratio, will serve as independent variable during which the flux control coefficient is being determined.

The following section provides the theoretical foundations to thermodynamics. This is followed by an overview of biochemical modeling and metabolic control analysis.


\section{Thermodynamics}
Energy in our universe is contained within a gradient, from complete order to complete disorder as described by the laws of thermodynamics. From the first and second laws, energy is known to be indestructible and finite, yet is capable of changing from one state to another. Within these constraints life as we know it have developed and thrived, as an intermediate stability within gradients. This leaves the probability that regulatory strategies of living systems have developed in such a way as be reflective of the underlying thermodynamic properties of the individual components involved.
\cite{Schneider1994, Prigogine1998}. 

Entropy ($S$), as the main driving force in thermodynamics, is said to give rise to the arrow of time  \cite{Lebowitz1994}. In chemistry, $S$ is further described in the diffusive movements of molecules. By state changes that proceed from regions of high order to regions of low order \cite{Roos2014}. In biochemical systems these states are further influenced by constraints on compartmental environments, as in vessels, cells and organelles \cite{Kramers1940, Burada2009}. Irrespective of constraints however, the underlying laws of thermodynamics hold and a change in state is invariably bound to a change in energy. This change can relate to a decrease in $S$ of the system itself, but ultimately leading to an increase in $S$ of some larger system. 

In a biochemical systems, such a change is quantified by the Gibbs free energy change ($\Delta G$). $\Delta G$ defines the change in energy state of a system, accounting for the change to internal energy ($\Delta H$), as well as change in entropy ($\Delta S$) and it's reliance on temperature ($T$) \cite{Job2016, LEFFLER1955}.

\begin{equation}\label{gibbsFree}
\Delta G = \Delta H - T \Delta S
\end{equation}

$\Delta G$ also yields insight into the direction of a reaction based on the current thermodynamic state. With a natural tendency towards a decrease in Gibbs free energy, reactions with a negative $\Delta G$ at current conditions, will proceed spontaneously without the addition of energy, termed exergonic. A positive $\Delta G$ indicates a reaction requiring the addition of energy in order to proceed, also termed an endergonic reaction.  With $\Delta G$  serving as indication of reaction direction, the standard Gibbs free energy ($\Delta G^0$) serves as reference of the equilibrium state. $\Delta G^0$ is defined by the standard state, in which a constant pressure of $1 atm$ a temperature of $298K$ a concentration of $1M$ for all species and a pH of 7 is maintained. Equation \ref{gibbsFree} becomes

\begin{equation}
\Delta G^0 = \Delta H^0 - 298 \times \Delta S^0  
\end{equation}

At the standard state, an equilibrium constant $K_{eq}$ for the reaction can be defined as

\begin{equation}
\Delta G^0 = -RT \ln K_{eq}  
\end{equation}

with $R$ being the ideal gas constant $8.3145$. One can therefore define a reaction in terms of direction and distance from equilibrium by taking account of $\Delta G^0$ in $\Delta G$ of the reaction with $\Gamma$ indicating the mass-action (product over substrate) ratio at the current state.

\begin{equation}\label{mass_action}
\Delta G = \Delta G^0 + RT \ln \Gamma
\end{equation}

From the above, it is clear that the equilibrium state at standard conditions, current metabolite state as well as thermodynamic constants relate to the change in Gibbs free energy of the reaction. However, the Gibbs free energy change as defined above does not give information about individual species contributions towards the systemic change.

At equilibrium for instance, the change in energy of products and substrates equal one another, resulting in a zero net change. Considering the unit, $kj/mol$, of Gibbs free energy at equilibrium, the energy per mol from each species contributes equaly to the formation energy of the other.

\begin{equation}
\Delta G_{substrates} = \Delta G_{products}
\end{equation}

It is important to note that such a state does not describe the equilibrium constant. unless observed at the standard condition as defined by $\Delta G^0$. 

Since biochemical systems consist of multi-molecular solutions, the general Gibbs free energy change can be adapted in order to more fully describe individual species contributions. As such, each species contributes a partial energy change towards the full Gibbs free energy change. This partial change per species unit is termed the chemical potential ($\mu $) of species ($i$). $\Delta H$  in equation \ref{gibbsFree} describes the change to internal energy, as being equal to the total internal species energy $U$ plus the product of the pressure $P$ and volume $V$.

\begin{equation}
H = U + PV
\end{equation}

$U$ can be rewritten in terms of the sum of chemical potentials $\mu$ of all species $i$ involved in the exchange of the number of units $N$, such that

\begin{equation}
H = \sum^{n}_{i=1}\mu_i \delta N_i + PV
\end{equation}

Substituting $H$ back into equation \ref{gibbsFree}, the Gibbs free energy change for the system is now defined in terms of individual species contributions.

\begin{equation}
\Delta G = \sum_{i=1}^{n}\mu_i N_i - T \Delta S
\end{equation}

Through the Gibbs free energy changes and chemical potentials, a description of the reaction direction as well as distance from equilibrium is quantifiable. However, the Gibbs free energy change does not yield information on the regulatory mechanisms or specific rates of reactions. These are tasks better handled by the construction of rate equations.

\section{Rate Equations}

Rate equations, constructed from rate laws and parameterised from experimental observations. This is done by defining the structure and overall regulatory properties of the system \cite{Teusink2000, Sorribas1995}. An example of a fundamental rate law, the law of mass action, handled in equation \ref{mass_action}, as described by \citeauthor{Waage1986} was derived from thermodynamic principles by \citeauthor{VantHoff1884}. This rate law have since been used as the basis for more generic rate equations \cite{VantHoff1884, Waage1986, Voit2015}. 

An example of a simple rate equation construct from the law of mass action describes the rate of change for $[A]$, where $A + B \longrightarrow C$. This can be defined as

\begin{equation}\label{rateOfChange}
\frac{d[A]}{dt} = -v_A = k[A]^\alpha[B]^\beta
\end{equation}

The rate of change of $[A]$ ($v_A$) is influenced by factors of metabolite concentration ($[A]$, $[B]$), reaction order ($\alpha$, $\beta$) and rate constants ($k$), in short termed the probability of interaction. Reaction orders in turn relate a reaction's sensitivity to metabolites, while the rate constants, as termed by the Arrhenius equation,

\begin{equation}\label{arenius}
k = Ae^{-\frac{E_A}{RT}}
\end{equation}

describes the dependence on temperature ($T$), activation energy ($E_A$), the gas constant ($R$) and a spatial frequency factor ($A$). From equation \ref{rateOfChange} one can examine the equilibrium condition, where no net change in concentration of $A$ is observed, such that

\begin{equation}
v_A = 0
\end{equation} 

An equilibrium solution is achieved as any of the metabolite concentrations approach zero. However, since reactions do not only proceed in one direction, an expansion on the example above is made. By defining the reaction proceeding reversibly such that $A + B \Longleftrightarrow  C$ the need for two separate reaction occurrences describing the net rate of change ($v$) becomes apparent. The reaction in the forward direction, defined as the consumption of $A$ and $B$, 

\begin{equation}\label{forwardReac}
    v_{forward}_A = k_A[A]^\alpha[B]^\beta + k_B[A]^\alpha[B]^\beta
\end{equation} 

and the reverse reaction, the production of $A$, defined as 

\begin{equation}\label{reverseReac}
v_{reverse} = k_{-A}[C]^\gamma
\end{equation}

Considering the net reaction rate 

\begin{equation}\label{netReac}
v_{net} = v_{forward} - v_{reverse}
\end{equation}

one can substitute equations \ref{forwardReac} and \ref{reverseReac} into equation \ref{netReac} yielding 

\begin{equation}
v_{net} = k_A[A]^\alpha[B]^\beta - k_{-A}[C]^\gamma
\end{equation}

with reactions at equilibrium having equal and opposing rate magnitudes.

Through the Haldane relationship, one is able to relate the equilibrium state to both the thermodynamic and kinetic properties by way of the mass action ratio and rate constants.

\begin{equation}
\frac{[C]_{eq}}{[A]_{eq}[B]_{eq}} = k_{eq} = \frac{k_{-A}}{k_A}
\end{equation}

Rate laws and equations, as described above, have formed the basis of quantifying experimental observations into a mathematical formalism of models. Biochemical models are constructed by linking together many of these rate equations in an ODE structure. This structure resembles a metabolic network of production and consumption rates leading to changes in metabolic concentrations \cite{Copeland2000, Hynne2001, Kell2006}. 

\section{Metabolic Control Analysis (\gls{mca})}
\gls{mca}, also referred to as sensitivity analysis or control theory, brought about advances in biological understanding in the form of a framework within which to incorporate both structural and kinetic information of a system as a whole. As a subset of sensitivity analysis, \gls{mca} is concerned with systematically quantifying effects upon global systems as brought about by alterations in local parameters. Effects of such alterations, also known as a perturbations, are investigated during what is known as the \gls{steady-state}. This state defines a dynamic equilibrium, whereby the first and last metabolites, known as pools and sinks, are held constant (buffered). In this state, alterations of local enzyme and metabolite concentrations give rise to resulting global flux and concentration changes. Dependant upon the type of perturbation study, the flux control-, concentration control- and elasticity co\"efficients are three response co\"efficients that can be derived \citep{Kacser1979, Kacser1968}. 

\subsection{Control co\"efficients and elasticities} 
The flux control co\"efficient ($C^{J}_{i}$), as the name implies, yields a quantitative description of the effect upon flux ($J$) that reaction ($i$) brought about via local enzyme concentration ($e_i$) perturbations.

\begin{equation}
C^{J}_{i}\approx\dfrac {\delta J}{J}/\dfrac {\delta e_{i}}{e_{i}}
\end{equation}

The second co\"efficient, describing the effect upon metabolite ($S$), given a change in enzyme ($e_{i}$) concentration, is termed the concentration control co\"efficient.

\begin{equation}
C^{S}_{i}\approx\dfrac{\delta S}{S}/\dfrac{\delta e_{i}}{e_{i}}
\end{equation}

The third co\"efficient, elasticity, relates a change in system parameter to the corresponding direct change in the rate of the reaction. For example, the effect of temperature, metabolite concentrations, inhibitors and $pH$. The elasticity co\"efficient is defined as.

\begin{equation}
\varepsilon^{i}_{P}\approx\dfrac{\delta v_{i}}{v_{i}}/\dfrac{\delta P}{P}
\end{equation}

Whilst the first two co\"efficients, respectively termed the flux control and concentration control co\"efficients, relate local parameter changes to global effects, the elasticity control co\"efficient relate local variable changes to corresponding reaction rate changes. Therefore elasticity determinations can be done in isolated enzyme kinetic reactions, as long as metabolite concentrations are kept at $in vivo$ steady sate concentrations. 

From these control co\"effiecients and elasticities two relationships, the connectivity and control theorems, were derived. These theorems were developed in order to; relate the relationships described above to one another as well as setting constraints upon the system within which a more systemic analysis can occur.

\subsection{Summation and Connectivity Theorems}

The summation theorem as described by \citeauthor{Kacser1979,Rapoport1974} defines boundaries of systems, setting constraints on the above mentioned coeficients, such that 

\begin{equation}\label{fluxSummation}
\sum_{i}C_{v_i}^J=1
\end{equation}

\begin{equation}\label{concSummation}
\sum_{i}C_{v_i}^S=0
\end{equation}

Equation \ref{fluxSummation} describes the distribution of flux control throughout the system, with each reaction sharing in part towards the whole of flux control for the pathway. Equation \ref{concSummation} describes the effect upon metabolite concentrations via enzyme perturbations as it is distributed throughout the system. In other words, as concentrations in one part increases with respect to parameter perturbations, concentrations in other parts will decrease in accordance.

Connectivity theorems in turn define the link between local enzyme kinetic properties and global pathway variables based on two basic assumptions. The first assumption is on the additive properties between the products of flux control and elasticity co\"effiecients.

\begin{equation}
\sum_{i}C_{i}^J\epsilon_S^i=0
\end{equation}

The second theorem involves the summation property of the products between concentration control and elasticity co\"effiecients. This theorem is subdivided into two equations, the first of which considers system species,($S_m$) differing to those of local species ($S_m$) while the second considers system species equal to local species \citep{Kacser1979, Westerhoff1984}.

\begin{equation}
\sum_{i}C_{i}^{S_n}\epsilon_{S_m}^i=0\quad \textrm{for}\quad m\ne n
\end{equation}

\begin{equation}
\sum_{i}C_{i}^{S_n}\epsilon_{S_m}^i=-1\quad \textrm{for}\quad m = n
\end{equation}

These relationships were further extended to symbolic matrix algebra methods \cite{Fell1992,Kacser1995,Ehlde1997,Hofmeyr2001}.


Investigations and understandings of biological processes could now occur in a less isolated \textit{i.e.} more holistic manner, as the concept of a rate-limiting step was replaced by one of shared control. \citeauthor{Kacser1968} illustrates the earliest intuition towards \gls{mca} on the basis of experimental investigations methods. The first of which investigates isolated portions a system with enzyme kinetic results largely driven by \textit{in vitro} experimental work from isolated enzymes, echoing the elasticity co\"efficient determinations described above \citep{Hynne2001, Boren2002}. The second approach maintains the system at a constant state by way of maintaining the beginning and end specific variables constant and one by one altering others and noting the effects observed, relating to flux and control co\"efficient determinations \citep{Kacser1968}.

The implication and applications of \gls{mca} have been far reaching. Bringing about new insights and developments to the domains of Biochemistry, Biomedicine, Biotechnology, Physiology, Pharmaceuticals, Genetics, Botany, Immunology, Agriculture and more \citep{Kacser1981, Sorribas1995, Holms1996, Cornish1999, Boren2002, Cascante2002, Olivier2004, Weselake2008}. With an application on such a wide spectrum of disciplines an approach with \gls{mca} as basis would prove useful in the creation of a generic function for multiple model control analysis.  

Many research activities have been geared towards discerning the so called "rate-limiting" step of a metabolic pathway, adopted from chemistry where the overall reaction rate is approximately determined by the slowest rate in the reaction chain. This concept serves as a useful tool in simplifying the mathematics involved in kinetic calculations of reactions. However, this oversimplification results in a loss of network accuracy when applied to more complex reaction pathways. The introduction of \gls{mca}, specifically the flux control summation theorem redefined the concept of the "rate-limiting" step, by defining the contribution of each reaction towards the distribution of control over the system as a whole. In the case of flux control, as opposed to having a single "rate-limiting" reaction controlling the flux through the entire system. Assuming at this point that the sum of the own-flux control co\"efficients across the system equals one, the system is said to obey the flux summation theorem \citep{Fell1992,Hofmeyr2001,Kacser1995,Ehlde1997}. This theorem served a a fundamental test towards model usability, described in chapter \ref{chp:3}.

\section{Biochemical Modeling}
Models are able to obtain degrees of accuracy and realism defined on an, as needed basis, by incorporating experimental observations at relevant physiological conditions. These observations are included through the use of rate equations and parameters, as is described 

The end result, ideally, is a model that details all of the above observations succinctly, elegantly and in a robust (reusable) manner. In practice however, the level of detail needed is largely determined on a case by case basis, as either a large depth of detail is not needed or not feasible within specified time frames. As such, little emphasis may have been placed on discerning all of the specific reactions within a pathway as reversible rate equations. This hybrid natures of varying degrees of detail will, in part, be addressed by the use of custom filtering functions as well as metabolic control analysis protocols, described in chapter \ref{Methodology}. The next section therefore views the relevance of \gls{mca} as a strategy towards discerning regulation properties of a reaction network.
